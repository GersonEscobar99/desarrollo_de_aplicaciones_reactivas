\documentclass[12pt]{article}
\usepackage[utf8]{inputenc}
\usepackage[spanish]{babel}
\usepackage{csquotes}
\usepackage[backend=biber,style=apa,sorting=nyt]{biblatex}
\usepackage{times}
\usepackage{setspace}
\usepackage{geometry}
\usepackage{titlesec}

\geometry{letterpaper, margin=1in}
\setstretch{2}
\addbibresource{referencias.bib}

\titleformat{\section}{\normalfont\Large\bfseries}{\thesection}{1em}{}

\title{\textbf{Buenas Prácticas para el Desarrollo de Aplicaciones Reactivas}}
\author{Gerson Escobar}
\date{\today}

\begin{document}

\maketitle

\section*{Resumen}
El desarrollo de aplicaciones reactivas ha ganado popularidad en los últimos años debido a su capacidad para manejar flujos de datos asincrónicos, mejorar la experiencia del usuario y optimizar el uso de recursos del sistema. Este documento presenta una recopilación de buenas prácticas clave para su implementación eficaz.

\section{Introducción}
Las aplicaciones reactivas se basan en los principios de la programación reactiva, que enfatizan la asincronía, el flujo de datos y la resiliencia (Jonas, 2017). Frameworks como Spring WebFlux y bibliotecas como RxJava o Project Reactor permiten implementar sistemas más responsivos y eficientes.

\section{Buenas Prácticas}

\subsection{1. Comprender el modelo de programación reactivo}
No basta con usar un framework reactivo; es necesario entender el paradigma. Esto incluye conocer los conceptos de flujo de datos, observables, backpressure y suscripción.

\subsection{2. Utilizar operadores correctamente}
RxJava, Reactor y otros proporcionan operadores como `map`, `flatMap`, `filter`, `buffer`, entre otros. Su uso correcto evita problemas de rendimiento y facilita el mantenimiento del código.

\subsection{3. Manejo adecuado de errores}
Implementar estrategias como `onErrorResume`, `retry`, `timeout` y `fallback` ayuda a construir sistemas resilientes que se recuperan ante fallas.

\subsection{4. Evitar bloqueos}
No se deben usar llamadas bloqueantes (`Thread.sleep()`, `blocking I/O`) dentro de flujos reactivos. Utiliza conectores reactivos como R2DBC o WebClient para mantener la no bloqueabilidad.

\subsection{5. Usar backpressure}
La presión inversa (backpressure) permite evitar la sobrecarga del sistema al controlar la velocidad de emisión y consumo de eventos.

\subsection{6. Monitoreo y trazabilidad}
Las aplicaciones reactivas son más difíciles de depurar. Por ello, se recomienda implementar logs estructurados, herramientas de trazabilidad como Zipkin o Sleuth, y métricas con Micrometer.

\subsection{7. Testeo de flujos}
Se deben usar bibliotecas como `StepVerifier` (Project Reactor) o `TestObserver` (RxJava) para probar el comportamiento de los flujos de manera controlada y reproducible.

\section{Conclusión}
El desarrollo reactivo ofrece grandes beneficios, pero su correcta implementación requiere un cambio de paradigma y disciplina en el uso de herramientas. Aplicar estas buenas prácticas contribuye a lograr sistemas robustos, escalables y eficientes.

\section{URL GIT}
\href{https://github.com/GersonEscobar99/buenas-practicas-de-aplicaciones-reactivas.git} 

\printbibliography

\end{document}
